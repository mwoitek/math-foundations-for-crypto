% Created 2024-06-24 Mon 21:42
% Intended LaTeX compiler: pdflatex
\documentclass[11pt]{article}
\usepackage[utf8]{inputenc}
\usepackage[T1]{fontenc}
\usepackage{graphicx}
\usepackage{longtable}
\usepackage{wrapfig}
\usepackage{rotating}
\usepackage[normalem]{ulem}
\usepackage{amsmath}
\usepackage{amssymb}
\usepackage{capt-of}
\usepackage{hyperref}
\usepackage[a4paper,left=1cm,right=1cm,top=1cm,bottom=1cm]{geometry}
\usepackage[american, english]{babel}
\usepackage{enumitem}
\usepackage{float}
\usepackage[sc]{mathpazo}
\linespread{1.05}
\renewcommand{\labelitemi}{$\rhd$}
\setlength\parindent{0pt}
\setlist[itemize]{leftmargin=*}
\setlist{nosep}
\newcommand{\Mod}{\mathrm{mod}\:}
\author{Marcio Woitek}
\date{}
\title{Integer Foundation}
\hypersetup{
 pdfauthor={Marcio Woitek},
 pdftitle={Integer Foundation},
 pdfkeywords={},
 pdfsubject={},
 pdfcreator={Emacs 29.3 (Org mode 9.8)}, 
 pdflang={English}}
\begin{document}

\maketitle
\thispagestyle{empty}
\pagestyle{empty}
\section*{Problem 1}
\label{sec:org629dde8}
\textbf{Answer: \(819,990\)}\\

\textbf{There's a problem with this question. All the options are wrong, i.e., there's
no pair of relatively prime numbers. Then I had to get the correct answer by
guessing.}\\

It's simple to exclude all the options:
\begin{itemize}
\item \(91,343\): Both of these numbers are divisible by 7. To see this, first
notice that \(91=70+21\). Then \(\frac{91}{7}=13\). Similarly, we can
write \(343=350-7\). Hence: \(\frac{343}{7}=49\). Since 91 and 343 have a
common prime factor of 7, they are not relatively prime.
\item \(819,990\): These numbers are divisible by 3. If we add up the digits of
819 we get 18, which is divisible by 3. Then 819 is divisible by 3. A similar
argument applies to 990. Therefore, 819 and 990 have 3 as a common factor, and
they're not relatively prime.
\item \(796,982\): These numbers are clearly even. Since they have a common factor
of 2, they're not relatively prime.
\item \(527,612\): In this case, we compute the \(\mathrm{gcd}\) of 527 and 612.
With the aid of the Euclidean algorithm, we obtain
\end{itemize}
\begin{center}
\begin{tabular}{|c|c|}
\hline
\(m\) & \(n\)\\
\hline
612 & 527\\
527 & 85\\
85 & 17\\
17 & 0\\
\hline
\end{tabular}
\end{center}
So we have \(\gcd(527,612)=17\). This result makes it very clear that the
numbers under consideration are not relatively prime.
\section*{Problem 2}
\label{sec:orgc6995e4}
\textbf{Answer: 62}\\

Applying the Euclidean algorithm, we get the following:
\begin{center}
\begin{tabular}{|c|c|}
\hline
\(m\) & \(n\)\\
\hline
992 & 930\\
930 & 62\\
62 & 0\\
\hline
\end{tabular}
\end{center}
Then \(\gcd(930,992)=62\).
\section*{Problem 3}
\label{sec:org1cb3142}
\textbf{Answer: \(-17,9\)}\\

To find the congruent pair, we need to test each pair by actually doing the
calculations. We're not going to present these calculations. But it turns out
that the correct answer is \(-17,9\). This result can be verified as follows:
\begin{equation}
-17\equiv -17+2\cdot 13\equiv 9\quad(\Mod 13),
\end{equation}
where we've used the fact that any multiple of 13 is congruent to 0 (\(\Mod 13\)).
\section*{Problem 4}
\label{sec:orgc3d7c5a}
\textbf{Answer: \(x\) is a multiple of half the modulus.}\\

Recall that \(a\) and \(b\) are congruent modulo \(N\) when their
difference \(a-b\) is divisible by \(N\). In this case, this difference is
\(x-(-x)=2x\). Since \(N\) divides \(2x\), there's an integer \(k\) such
that the following equation holds:
\begin{equation}
\frac{2x}{N}=k.
\end{equation}
Solving this equation for \(x\), we get
\begin{equation}
x=k\frac{N}{2}.
\end{equation}
Therefore, \(x\) is a multiple of half the modulus.
\section*{Problem 5}
\label{sec:org9f8cd24}
\textbf{Answer: \(x\) must be relatively prime to \(N\).}
\section*{Problem 6}
\label{sec:orgbea2fab}
\textbf{Answer: \(O(\log n)\) (where \(n\) is the smaller number).}
\section*{Problem 7}
\label{sec:org96a9e0e}
\textbf{Answer: 3}\\

We'll solve this problem by using the general version of the extended Euclidean
algorithm. This algorithm will yield a solution to the equation
\begin{equation}
16x+47y=\gcd(16,47)=1.
\end{equation}
The table below contains all the results returned by the Euclidean algorithm.
\begin{center}
\begin{tabular}{|c|c|c|c|c|c|c|c|}
\hline
\(m\) & \(n\) & \(q\) & \(r\) & \(s\) & \(t\) & \(\hat{s}\) & \(\hat{t}\)\\
\hline
47 & 16 & 2 & 15 & 1 & 0 & 0 & 1\\
16 & 15 & 1 & 1 & 0 & 1 & 1 & -2\\
15 & 1 & 15 & 0 & 1 & -2 & -1 & 3\\
\hline
\end{tabular}
\end{center}
The desired solution is in the last two columns of the last row: \(x=3\) and
\(y=-1\). Hence:
\begin{equation}
16\cdot 3+47\cdot(-1)=1.
\end{equation}
Performing the modulo operation for both sides of this equation, we get
\begin{equation}
16\cdot 3\equiv 1\quad(\Mod 47).
\end{equation}
Clearly, in this case we have \(16^{-1}=3\).
\section*{Problem 8}
\label{sec:org81d53da}
\textbf{Answer: -1}\\

We'll solve this problem by using the general version of the extended Euclidean
algorithm. This algorithm will yield a solution to the equation
\begin{equation}
219x+220y=\gcd(219,220)=1.
\end{equation}
The table below contains all the results returned by the Euclidean algorithm.
\begin{center}
\begin{tabular}{|c|c|c|c|c|c|c|c|}
\hline
\(m\) & \(n\) & \(q\) & \(r\) & \(s\) & \(t\) & \(\hat{s}\) & \(\hat{t}\)\\
\hline
220 & 219 & 1 & 1 & 1 & 0 & 0 & 1\\
219 & 1 & 219 & 0 & 0 & 1 & 1 & -1\\
\hline
\end{tabular}
\end{center}
The desired solution is in the last two columns of the last row: \(x=-1\) and
\(y=1\). Hence:
\begin{equation}
219\cdot(-1)+220\cdot 1=1.
\end{equation}
Performing the modulo operation for both sides of this equation, we get
\begin{equation}
219\cdot(-1)\equiv 1\quad(\Mod 220).
\end{equation}
Clearly, in this case we have \(219^{-1}=-1\).
\end{document}
