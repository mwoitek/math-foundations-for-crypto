% Created 2024-06-26 Wed 23:28
% Intended LaTeX compiler: pdflatex
\documentclass[11pt]{article}
\usepackage[utf8]{inputenc}
\usepackage[T1]{fontenc}
\usepackage{graphicx}
\usepackage{longtable}
\usepackage{wrapfig}
\usepackage{rotating}
\usepackage[normalem]{ulem}
\usepackage{amsmath}
\usepackage{amssymb}
\usepackage{capt-of}
\usepackage{hyperref}
\usepackage[a4paper,left=1cm,right=1cm,top=1cm,bottom=1cm]{geometry}
\usepackage[american, english]{babel}
\usepackage{enumitem}
\usepackage{float}
\usepackage[sc]{mathpazo}
\linespread{1.05}
\renewcommand{\labelitemi}{$\rhd$}
\setlength\parindent{0pt}
\setlist[itemize]{leftmargin=*}
\setlist{nosep}
\newcommand{\Mod}{\mathrm{mod}\:}
\author{Marcio Woitek}
\date{}
\title{Course Project}
\hypersetup{
 pdfauthor={Marcio Woitek},
 pdftitle={Course Project},
 pdfkeywords={},
 pdfsubject={},
 pdfcreator={Emacs 29.4 (Org mode 9.8)}, 
 pdflang={English}}
\begin{document}

\maketitle
\thispagestyle{empty}
\pagestyle{empty}
\section*{Problem 1}
\label{sec:org42cf67b}
\textbf{Answer:} 115\\

As explained, the ciphertext can be computed as follows:
\begin{equation}
c=m^e\:\Mod n.
\end{equation}
In this case, we have \(m=15\), \(e=7\) and \(n=143\). Hence:
\begin{equation}
c=15^7\:\Mod 143=115.
\end{equation}
\section*{Problem 2}
\label{sec:orgc7fb934}
\textbf{Answer:}
\begin{itemize}
\item 2
\item 61
\item 461\\
\end{itemize}

215 is not prime, since it is clearly divisible by 5.
\section*{Problem 3}
\label{sec:org3492d2d}
\textbf{Answer:}
\begin{itemize}
\item \(122=2\cdot 61\)
\item \(922=2\cdot 461\)\\
\end{itemize}

The other two integers are the product of three prime numbers:
\(13115=5\cdot 43\cdot 61\), \(99115=5\cdot 43\cdot 461\).
\section*{Problem 4}
\label{sec:org05ba776}
\textbf{Answer:} 460

\begin{equation}
\varphi(922)=\varphi(2\cdot 461)=(2-1)(461-1)=460
\end{equation}
\end{document}
