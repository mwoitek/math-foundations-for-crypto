% Created 2024-06-25 Tue 19:40
% Intended LaTeX compiler: pdflatex
\documentclass[11pt]{article}
\usepackage[utf8]{inputenc}
\usepackage[T1]{fontenc}
\usepackage{graphicx}
\usepackage{longtable}
\usepackage{wrapfig}
\usepackage{rotating}
\usepackage[normalem]{ulem}
\usepackage{amsmath}
\usepackage{amssymb}
\usepackage{capt-of}
\usepackage{hyperref}
\usepackage[a4paper,left=1cm,right=1cm,top=1cm,bottom=1cm]{geometry}
\usepackage[american, english]{babel}
\usepackage{enumitem}
\usepackage{float}
\usepackage[sc]{mathpazo}
\linespread{1.05}
\renewcommand{\labelitemi}{$\rhd$}
\setlength\parindent{0pt}
\setlist[itemize]{leftmargin=*}
\setlist{nosep}
\newcommand{\Mod}{\mathrm{mod}\:}
\author{Marcio Woitek}
\date{}
\title{Modular Exponentiation}
\hypersetup{
 pdfauthor={Marcio Woitek},
 pdftitle={Modular Exponentiation},
 pdfkeywords={},
 pdfsubject={},
 pdfcreator={Emacs 29.3 (Org mode 9.8)}, 
 pdflang={English}}
\begin{document}

\maketitle
\thispagestyle{empty}
\pagestyle{empty}
\section*{Problem 1}
\label{sec:org6314d71}
\textbf{Answer:} \(47^{69}\:\Mod 143=125\)
\section*{Problem 2}
\label{sec:org1418c7d}
\textbf{Answer:} \(15^{15}\:\Mod 14=15\)\\

We begin by reducing the exponent. The first step is to compute the totient
function for 14:
\begin{equation}
\varphi(14)=\varphi(2\cdot 7)=(2-1)(7-1)=6.
\end{equation}
Then, since 14 and 15 are relatively prime, we can write
\begin{equation}
15^{15}\equiv 15^3\quad(\Mod 14).
\end{equation}
Next, to reduce the third power of 15, we calculate the square of this number
modulo 14:
\begin{equation}
15^2\equiv 15\cdot 15\equiv 1\cdot 1\equiv 1\quad(\Mod 14),
\end{equation}
where we've used the fact that 15 is congruent to 1 \(\Mod 14\). Hence:
\begin{equation}
15^{15}\equiv 15\cdot 15^2\equiv 15\quad(\Mod 14).
\end{equation}
We could reduce the above power one more time. However, we stop here since 15 is
one of the options.
\section*{Problem 3}
\label{sec:orgaa166f9}
\textbf{Answer:} The number of positive integers less than and relatively prime to \(N\).
\section*{Problem 4}
\label{sec:org9f22119}
\textbf{Answer:} \(\varphi(p)=p-1\)
\section*{Problem 5}
\label{sec:orgc9c324f}
\textbf{Answer:} \(\varphi(2717)=2160\)\\

First, notice that 2717 can be factored as follows: \(2717=11\cdot 13\cdot 19\).
Then, by using the multiplicative property of \(\varphi\), we can write
\begin{equation}
\varphi(2717)=\varphi(11\cdot 13\cdot 19)=(11-1)(13-1)(19-1)=2160.
\end{equation}
\section*{Problem 6}
\label{sec:orge3a315a}
\textbf{Answer:} \(\varphi(3^4)=54\)\\

The formula for the totient of the \(k\)-th power of a prime \(p\) is
\begin{equation}
\varphi(p^k)=p^{k-1}(p-1).
\end{equation}
Applying this result for \(p=3\) and \(k=4\) yields
\begin{equation}
\varphi(3^4)=3^3(3-1)=27\cdot 2=54.
\end{equation}
\section*{Problem 7}
\label{sec:org3192587}
\textbf{Answer:} If the two numbers are relatively prime.
\section*{Problem 8}
\label{sec:org1b1b07c}
\textbf{Answer:} Nothing reliable.
\section*{Problem 9}
\label{sec:org0b55b58}
\textbf{Answer:} It may or may not exist to other bases.
\section*{Problem 10}
\label{sec:org0305ec0}
\textbf{Answer:} A primitive root generates all numbers relatively prime to the
modulus.
\end{document}
